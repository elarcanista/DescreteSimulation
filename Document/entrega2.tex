\input{wsc17style.tex}

\documentclass{wscpaperproc}

\usepackage{latexsym}
\usepackage{caption}
\usepackage{graphicx}
\usepackage{subcaption}
\usepackage[utf8]{inputenc}
\usepackage[table]{xcolor}

\begin{document}

\WSCpagesetup{Ortega and Salgado}

\title{QUEUE ANALYSIS FOR A SUPERCOMPUTING CENTER}

\author{Andrés Felipe Ortega Montoya\\ [12pt]
aortega7@eafit.edu.co\\
Ingeniería Matemática\\
Universidad EAFIT\\
\and
Alejandro Salgado Gómez\\[12pt]
asalgad2@eafit.edu.co\\
Ingeniería Matemática\\
Universidad EAFIT\\
}

\maketitle

\section{Conceptual modeling}

\subsection{Problem definition}

The increment of data and problem sizes in the last couple of years has
triggered an unprecedented growth in the necessity of computational
power. This necessity has induced an increase in the utilization of
elements like supercomputers.

Due to the high amount of tasks a supercomputer has to process
it has become of great importance to device systems to control the
order in which resources are assigned to jobs. However due to how
varied the amount of time each of these can take to compute, it can
constitute a problem to provide reasonable waiting times for all of its
users.

This situation produce a decrement in the quality of service
due to the high demand of the resources that a computing center offers.
For this reason the ideas and solutions that the discrete simulation can provide
are specially useful for this type of establishments, mainly because of
the improvements that can be made in terms of attention time and resource
utilization.

\subsection{Objectives}

\subsubsection{Model purpose}

The current model aims to present a simplified version of the the real system
in order to examine its behavior. This with the intention of understanding which
possible set of causes explains the current waiting times and  if possible determine
modifications that can improve the provided service for most of the user population.

\subsubsection{Specific objectives}

\begin{itemize}
    \item Have a discrete event modeling aproach of a real system to study the phenomenon
    \item Create an abstraction in order to better understand
          the process that characterized the real system.
    \item Analyze the behavior of the entities in each of the steps of the process.
    \item If possible, suggest ideas to optimize the performance of the actual system.
\end{itemize}

\subsubsection{Project objective}

This project aims to model the Scientific Computing Center APOLO as a
discrete event simulation. The main objective is to experiment the
application of the theory addressed in the course by working with a real
system. The intention is to analyze the occupation process of the
supercomputer and if it is possible, suggest alternatives to improve
the performance of the computing center in terms of attention to users
and resource utilization.

\subsection{Inputs and outputs}

\subsection{Contents}

\subsubsection{Scope}
\subsubsection{Detail level}
\subsubsection{Assumptions}

Apolo has a single queue with exponential distribution (lambda = 35)

\subsubsection{Simplifications}

Everything, we are basically modeling a single queue with exponential distribution

\section{State of the art}

\section{Data}

\section{Implementation}

\section{Verification and validation}

\section{Experimentation}

\subsection{Results}

\subsubsection{model nature} Salvaje y despiadado (Solo quiere ver el mundo arder)
\subsubsection{Output nature}
\subsubsection{Initial bias}
\subsubsection{Number of executions}
\subsubsection{Solution space}

\subsection{Sensitivity analysis}

\section{Conclusions}

\begin{itemize}
    \item This model is awesome! \cite{slurm}
\end{itemize}

\bibliographystyle{wsc}
\bibliography{entrega2}

\end{document}
